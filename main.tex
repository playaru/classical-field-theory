%!TEX program = xelatex
% !TEX root = main.tex
\documentclass[12pt,a4paper]{article}

% Packages
\usepackage[brazilian]{babel}           % Fix Latex dummy fonts
\usepackage{amsmath,amssymb,amsfonts,mathtools,amsthm} % Math
\usepackage{physics}               % Handy shorthand for derivatives, bras/kets, etc.
\usepackage{tensor}                % Indexed tensors
\usepackage{authblk}               % For clean multiple-author formatting
\usepackage{hyperref}              % Clickable links in PDF
\usepackage[margin=2.5cm]{geometry}
\usepackage{bookmark}              % Smarter PDF bookmark
\usepackage{graphicx}              % For figures
\usepackage{fancyhdr}              % Custom headers/footers
\usepackage{enumitem}              % Better control of lists/enumeration
\usepackage{xcolor}                % Color support
\usepackage{bm}                    % Bold math symbols
\usepackage{tikz}                  % For diagrams (optional but powerful)
\usepackage{cleveref}              % Smart cross-referencing (works well with hyperref)
\usepackage{titlesec}              % Customize section headings
\usepackage{titling}               % More control over the title
\usepackage{fontspec}
\usepackage{lettrine}
\usepackage{csquotes}
\usepackage{footnote}
\usepackage{manyfoot}
\usepackage{cancel}

%bib
\usepackage[backend=biber, style=numeric-comp, sorting=none]{biblatex}
\addbibresource{refs.bib}

%-------------------------------------Setup Visual---------------------------------------------%
\setlength{\parskip}{0.5em}       % Space between paragraphs
\setlength{\parindent}{0pt}       % No indentation
\pagestyle{fancy}
\fancyhf{}
\rhead{Notas em Teoria Clássica de Campos}
\lhead{GFTinho}
\cfoot{\thepage}

% Títulos, Subtítulos, etc.
\titleformat{\section}{\large\bfseries}{\thesection.}{1em}{}
\titleformat{\subsection}{\normalsize\bfseries}{\thesubsection.}{1em}{}
\titleformat{name=\section, numberless}
  {\large\bfseries\centering} % Formato do título: grande, negrito e centralizado
  {}                         % Rótulo (não usado em \section*)
  {0pt}                      % Espaço entre rótulo e título
  {}                         % Código a ser inserido antes do título


% Notas de Rodapé
\DeclareNewFootnote{B}[fnsymbol]


% --- CONFIGURAÇÃO DA FONTE CHIQUE (MÉTODO LOCAL) ---
% Usamos a opção "Path" para indicar a pasta onde a fonte está.
% O comando agora procura pelo NOME DO ARQUIVO, não pelo nome da família.
\newfontfamily{\capitular}[
  Path = ./,  % Indica a subpasta.
  Extension = .ttf % Opcional, mas boa prática.
]{Woodcut1}
%---------------------------------------------------------------------------------------------%

\begin{document}

\begin{titlepage}

    \newcommand{\HRule}{\rule{\linewidth}{0.5mm}} % Defines a new command for the horizontal lines, change thickness here
    
    \center % Center everything on the page
    %----------------------------------------------------------------------------------------
    %	HEADING SECTIONS
    %----------------------------------------------------------------------------------------
    \textsc{\LARGE Universidade Federal de Minas Gerais}\\[1.5cm] % Name of your university/college
    \textsc{\Large Programa de Graduação em Física}\\[0.5cm] % Major heading such as course name
    \textsc{\large ICEX}\\[0.5cm] % Minor heading such as course title
    \vfill
    %----------------------------------------------------------------------------------------
    %	TITLE SECTION
    %----------------------------------------------------------------------------------------
    \HRule \\[0.4cm]
    { \huge \bfseries Notas em Teoria Clássica de Campos}\\[0.4cm] % Title of your document
    \HRule \\[1.5cm]
    %----------------------------------------------------------------------------------------
    %	AUTHOR SECTION
    %----------------------------------------------------------------------------------------
    \begin{minipage}{0.4\textwidth}
    \begin{flushleft} \large
    \emph{Autores:}\\
    Álvaro \quad Cássia \quad Henrique \quad João Monteiro \quad Mariana \quad Matheus \quad Paulo Vitor % Your names
    \end{flushleft}
    \end{minipage}
    ~
    \begin{minipage}{0.4\textwidth}
    \begin{flushright} \large
    \emph{Professor:} \\
    Filipe \textsc{Menezes} 
    \end{flushright}
    \end{minipage}\\[2cm]
    \vspace*{6cm}
    %----------------------------------------------------------------------------------------
    %	DATE SECTION
    %----------------------------------------------------------------------------------------
    {\large \today}\\[2cm] 
    %----------------------------------------------------------------------------------------
    %	LOGO SECTION
    %----------------------------------------------------------------------------------------
    %\includegraphics[width = 0.3\linewidth]{principal_ufmg.jpg}
    %----------------------------------------------------------------------------------------
\end{titlepage}

\newpage

\section*{Convenções e Notação}\label{Conv_e_Not}
\addcontentsline{toc}{section}{Convenções e Notação}


\begin{description}[
    font=\bfseries,       % Deixa os termos em negrito
    style=unboxed,          % Alinha a descrição de forma limpa
    leftmargin=4.5cm,       % Define um espaço de 4.5cm para os termos
    labelsep=1cm            % Define um espaço de 1cm entre o termo e a descrição
]
    \item[Soma de Einstein] \hspace{0.2cm}$\displaystyle x_{\mu}y^{\mu} \equiv \sum_{\mu=0}^{3}x_{\mu}y^{\mu}$
    
    \item[Assinatura] \hspace{1.6cm}(-,+,+,+)
    
    \item[Unidades Naturais] $c = \epsilon_0 = \mu_0 = 1$.
    
    \item[Índices] \hspace{2.4cm}Índices gregos: 0,1,2,3; Índices latinos: 1,2,3.
    \item[Derivadas Parciais] $\displaystyle \partial_{\mu} \equiv \pdv{}{x^{\mu}}$ 
  \end{description}

\newpage

\tableofcontents % Gera o sumário
\newpage % Começa o conteúdo na página seguinte

\section{Relatividade Especial, Notação de Índices e Cálculo Tensorial}

\lettrine[lines=3, nindent=0.1em]{\capitular\addfontfeature{Color=AA0000}C}{omeçar} os estudos
em Teoria Clássica de Campos com Relatividade Especial, além de ser usual nos livros mais adotados
em cursos univesitários, é fundamental para a construção da base conceitual e do \textit{framework}
matemático necessário no tratamento de campos clássicos para as interações fundamentais.
Talvez o principal e mais famoso exemplo de um campo clássico desse tipo é o campo eletromagnético, que,
em sua natureza, é governado por leis físicas que são covariantes sob Transformações de Lorentz
(i.e., são as mesmas após realizarmos mudanças de coordenadas entre referenciais inerciais).
Começaremos agora a introduzir os conceitos mais importantes de relatividade especial para o tratamento de
campos clássicos fundamentais.

\subsection{Postulados da Relatividade Especial e Espaço-Tempo}
Enunciados por Einstein em 1905, ao tratar, justamente,
de questões fundamentais envolvendo o campo eletromagnético\cite{einstein1905}, os postulados da relatividade
especial podem ser adaptados em linguagem moderna da seguinte forma:
\begin{enumerate}[label=\Roman*.]
  \item As leis da Física são as mesmas para todos referenciais inerciais.
  \item A velocidade da luz é finita, com valor definido $c$ para todo e qualquer referencial.
\end{enumerate}

Estes são os alicerces a partir dos quais construiremos toda a teoria. Usaremos principalmente o postulado (I)
para nos guiar a respeito do estabelecimento de qualquer framework matématico que viermos a desenvolver. Para
entender a motivação por trás dos postulados, vale conferir o primeiro parágrafo em \cite{einstein1905}.

Devido aos postulados, surge de forma natural um tratamento equivalente de espaço e tempo na relatividade
especial, antes vistos como dois entes totalmente separados e independentes. Em outras palavras, devido à demanda
de que as leis da Física sejam covariantes por Lorentz, acabamos, inevitavelmente, por tratar espaço e tempo de
forma equivalente, fundindo-os no mesmo objeto que denominamos de \textbf{espaço-tempo}. Cada ponto desse
espaço-tempo, assinalado pelas coordenadas $(t,\vec{x})$, ou $(t,r,\theta,\varphi)$, ou qualquer outro sistema genérico de
coordenadas é chamado de \textbf{evento} nesse espaço-tempo.

Nessa nova forma de tratar o espaço e o tempo, de modo que as leis Físicas não mudem sob mudanças de referenciais
inerciais, acabamos introduzindo uma \textbf{nova métrica}, isto é, uma nova forma de \enquote{medir distâncias} nesse novo
\enquote{espaço} construído a partir dessa \enquote{fusão} entre espaço e tempo. Essa nova métrica é dada
por $\mathrm{ds^2}$, que não é para ser entendida como \enquote{um número pequeno elevado ao quadrado}, mas sim 
como um objeto por si só, que dita a medida das \enquote{distâncias} nesse espaço-tempo.
Em coordenadas cartesianas para um espaço-tempo plano, sua métrica é dada por:
\begin{equation*}
  ds^2 = -dt^2 + d\vec{x}^2
\end{equation*}
onde $d\vec{x}^2$ representa $dx^2 + dy^2 + dz^2$ para o espaço tridimensional (notação boba). Novamente,
$\mathrm{dt}$ e $\mathrm{dx}$ não representam \enquote{infinitesimais}, mas podem ser entendidos como tal num primeiro momento;
mais adiante, pretendemos esclarecer o real significado matemático que esses objetos carregam, quando formos
descrever a métrica de forma tensorial\footnote{Uma generalização de matrizes, ou melhor, de vetores; nada assustador.}.

\subsection{Invariância do Intervalo}

Em espaços euclidianos, integramos um \textbf{elemento de arco}\footnoteB{Obtido a partir da métrica euclidiana: $ds = \sqrt{dx^2 + dy^2}$ , que é a raiz quadrada da métrica euclidiana.} $\mathrm{ds}$
ao longo de uma curva $C$ para obtermos o \textbf{comprimento de arco} dessa curva. De forma completamente análoga,
isso também vale para espaços-tempos\footnote{Mais adiante mostraremos como obter o elemento de arco a partir da métrica em espaços-tempos; adiantamos que não é tão trivial quanto em espaços euclidianos, onde basta tirar a raiz quadrada da métrica. O espaço-tempo guarda algumas nuâncias...}:
\begin{equation}
  S \equiv \int_{C} ds .
\end{equation}

E as semelhanças não param por aí: assim como distâncias são preservadas em espaços euclidianos ao mudarmos de sistema de coordenadas, o comprimento
de um arco no espaço-tempo é o mesmo, independente do sistema de coordenadas adotado, ou seja:
\begin{equation}
  S = S' 
  \label{2}
\end{equation}
para a mesma curva $C$ nesse espaço-tempo, e para quaisquer sistemas de coordenadas que se relacionam por
transformações de Lorentz (\enquote{com linha} e \enquote{sem linha} denotam referenciais diferentes). 

Para provar a afirmação em (\ref{2}), é suficiente mostrar que o elemento
de arco $ds$ é o mesmo que seu correspondente $ds'$ no outro referencial em cada ponto da curva. Mas como temos posse apenas da métrica,
e sabemos que, de uma forma padronizada para ambos os referenciais, os elementos de arco $ds$ e $ds'$ são obtidos a partir das suas respectivas métricas, basta mostrar que:
\begin{equation*}
  ds^2 = ds^{\prime \; 2} \Rightarrow ds = ds' \Rightarrow S = S'
\end{equation*}
Sendo essa uma condição suficiente para provar a invariância do arco de uma curva situada no
espaço-tempo. 
Considere então a métrica descrita pelos referenciais sem linha e com linha:
\begin{equation}
  ds^2 = -dt^2 + dx^2 \;\;\;\;;\;\;\;\;\; ds^{\prime\; 2} = -dt^{\prime\; 2} + dx^{\prime\; 2}.
  \label{eq:intervalos}
\end{equation}
Para encontrar a relação entre as métricas, seria bom obter as relações entre os sistemas de coordenadas (t,x) e (t',x'),
pois conseguiremos relacionar $dt^{\prime \; 2}$ com $dt^2$ e $dx^{\prime \; 2}$ com $dx^2$.

Para isso, vamos   usar a forma explícita da transformação de Lorentz que faz a passagem de
um referencial que observa os eventos do espaço-tempo\footnote{Tomamos um espaço-tempo bidimensional (uma coordenada temporal, outra espacial) por simplicidade.} com as coordenadas $\mathrm{(t,x)}$ para um outro que
os observa com as coordenadas $\mathrm{(t',x')}$.
\begin{equation*}
  \Lambda = \begin{pmatrix*} \mathrm{\gamma} &  \mathrm{-\gamma v}\\
                          \mathrm{-\gamma v} & \mathrm{\gamma}  \end{pmatrix*}
\end{equation*}
onde $\mathrm{v}$ é o módulo da velocidade relativa entre os dois referenciais, e $\gamma$ é o fator de Lorentz, dado por
$\gamma = \mathrm{(1-v^2)^{-1/2}}$. Em notação de índices, podemos expressar a relação entre os dois sistemas 
de coordenadas na forma:
\begin{equation*}
x^{\mu'} = \Lambda^{\mu'}_{\mu} x^{\mu}
\end{equation*}
que é basicamente a atuação da matriz de $\Lambda$ nos vetores coluna coordenados escrita em termos das componentes (c.f. \ref{Conv_e_Not}{§ Convenções e Notação}, p. 1).

Assim, obtemos:
\[
  \left.
  \begin{aligned}
    &   t' = \gamma(t - \mathrm{v}x) \\
    &   x' = \gamma(-\mathrm{v}t + x) 
  \end{aligned}\;\;
  \right\rangle\;\;\;\;
  \begin{aligned}
    & dt' = \gamma(dt - \mathrm{v}dx)\\
    & dx' = \gamma(-\mathrm{v}dt +dx)
  \end{aligned}\\\\
  \therefore 
  \left\{\;
  \begin{aligned}
    & dt^{\prime\;2} = \gamma^2(dt^2 + \mathrm{v}^2dx^2 - 2\mathrm{v}dtdx)\\
    & dx^{\prime\;2} = \gamma^2(\mathrm{v}^2dt^2 + dx^2 - 2\mathrm{v}dtdx)
  \end{aligned}\right.
\]
Inserindo essas expressões na equação (\ref{eq:intervalos}) para o referencial com linha, concluímos:
\begin{proof}[]
  \begin{align}
    ds^{\prime\; 2} &= -dt^{\prime\; 2} + dx^{\prime\; 2} \nonumber \\
    &= -\gamma^2(dt^2 + \mathrm{v}^2dx^2 - \cancel{2\mathrm{v}dtdx}) + \gamma^2(\mathrm{v}^2dt^2 + dx^2 - \cancel{2\mathrm{v}dtdx}) \nonumber \\
    &=  \underbrace{\gamma^2 (1-\mathrm{v}^2)}_{= 1}(-dt^2 + dx^2) \nonumber\\
    &= -dt^2 + dx^2 \nonumber \\
    &= ds^2 \qedhere
  \end{align}
\end{proof}
Com isso provamos que a métrica é idêntica para referenciais que se relacionam por transformações de
Lorentz\footnote{O termo \textit{referenciais} é usado, muitas vezes, de forma equivalente a \textit{sistemas de coordenadas}, pois o segundo é a descrição matemática do conceito físico do primeiro em relatividade.},
o que é suficiente para provar a afirmação em \ref{eq:intervalos}.

Por último, no caso particular em que essa curva $C$ é uma reta ligando dois pontos fixos $A = (t_A,x_A)$ e $B=(t_B,x_B)$ no espaço-tempo,
escrevemos:

\begin{equation}
  {\Delta S_{AB}}^2 = -(t_B - t_A)^2 + (x_B - x_A)^2 = -\Delta t^2 + \Delta x^2
\end{equation}
e chamamos essa quantidade de \textbf{intervalo} entre os dois eventos $A$ e $B$. Como foi provado que, para uma mesma curva
$C$ ligando dois pontos, o comprimento de arco é invariante por Lorentz, fica evidente que o intervalo entre
dois eventos também é invariante por Lorentz, por se tratar de um caso particular do mais geral já que foi provado.
\section{Introdução ao Princípio Variacional}

\section{Lagrangiana do Campo Eletromagnético}

\newpage
\printbibliography
\end{document}
