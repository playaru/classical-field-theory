%!TEX program = xelatex
% !TEX root = main.tex
\documentclass[12pt,a4paper]{article}

% Packages
%\usepackage[utf8]{inputenc}        % Padrão
\usepackage[T1]{fontenc}           % Padrão tbm
\usepackage[brazilian]{babel}
\usepackage{lmodern}               % Fix Latex dummy fonts
\usepackage{amsmath,amssymb,amsfonts,mathtools} % Math
\usepackage{physics}               % Handy shorthand for derivatives, bras/kets, etc.
\usepackage{tensor}                % Indexed tensors
\usepackage{authblk}               % For clean multiple-author formatting
\usepackage{hyperref}              % Clickable links in PDF
\usepackage[margin=2.5cm]{geometry}
\usepackage{bookmark}              % Smarter PDF bookmark
\usepackage{graphicx}              % For figures
\usepackage{fancyhdr}              % Custom headers/footers
\usepackage{enumitem}              % Better control of lists/enumeration
\usepackage{xcolor}                % Color support
\usepackage{bm}                    % Bold math symbols
\usepackage{tikz}                  % For diagrams (optional but powerful)
\usepackage{cleveref}              % Smart cross-referencing (works well with hyperref)
\usepackage{titlesec}              % Customize section headings
\usepackage{titling}               % More control over the title
\usepackage{fontspec}
\usepackage{lettrine}

%-------------------------------------Setup Visual---------------------------------------------%
\setlength{\parskip}{0.5em}       % Space between paragraphs
\setlength{\parindent}{0pt}       % No indentation
\pagestyle{fancy}
\fancyhf{}
\rhead{Notas em Teoria Clássica de Campos}
\lhead{GFTinho}
\cfoot{\thepage}
\titleformat{\section}{\large\bfseries}{\thesection.}{1em}{}
\titleformat{\subsection}{\normalsize\bfseries}{\thesubsection}{1em}{}

% --- CONFIGURAÇÃO DA FONTE ORNAMENTADA (MÉTODO LOCAL) ---
% Usamos a opção "Path" para indicar a pasta onde a fonte está.
% O comando agora procura pelo NOME DO ARQUIVO, não pelo nome da família.
\newfontfamily{\capitular}[
  Path = ./,  %<-- AQUI ESTÁ O SEGREDO! Indica a subpasta.
  Extension = .ttf %<-- Opcional, mas boa prática.
]{Woodcut1}


\begin{document}

\begin{titlepage}

    \newcommand{\HRule}{\rule{\linewidth}{0.5mm}} % Defines a new command for the horizontal lines, change thickness here
    
    \center % Center everything on the page
    %----------------------------------------------------------------------------------------
    %	HEADING SECTIONS
    %----------------------------------------------------------------------------------------
    \textsc{\LARGE Universidade Federal de Minas Gerais}\\[1.5cm] % Name of your university/college
    \textsc{\Large Programa de Graduação em Física}\\[0.5cm] % Major heading such as course name
    \textsc{\large ICEX}\\[0.5cm] % Minor heading such as course title
    \vfill
    %----------------------------------------------------------------------------------------
    %	TITLE SECTION
    %----------------------------------------------------------------------------------------
    \HRule \\[0.4cm]
    { \huge \bfseries Notas em Teoria Clássica de Campos}\\[0.4cm] % Title of your document
    \HRule \\[1.5cm]
    %----------------------------------------------------------------------------------------
    %	AUTHOR SECTION
    %----------------------------------------------------------------------------------------
    \begin{minipage}{0.4\textwidth}
    \begin{flushleft} \large
    \emph{Autores:}\\
    Álvaro \quad Cássia \quad Henrique \quad João Monteiro \quad Mariana \quad Matheus \quad Paulo Vitor % Your names
    \end{flushleft}
    \end{minipage}
    ~
    \begin{minipage}{0.4\textwidth}
    \begin{flushright} \large
    \emph{Professor:} \\
    Filipe \textsc{Menezes} 
    \end{flushright}
    \end{minipage}\\[2cm]
    \vspace*{6cm}
    %----------------------------------------------------------------------------------------
    %	DATE SECTION
    %----------------------------------------------------------------------------------------
    {\large \today}\\[2cm] 
    %----------------------------------------------------------------------------------------
    %	LOGO SECTION
    %----------------------------------------------------------------------------------------
    %\includegraphics[width = 0.3\linewidth]{principal_ufmg.jpg}
    %----------------------------------------------------------------------------------------
\end{titlepage}

\tableofcontents % Gera o sumário
\newpage % Começa o conteúdo na página seguinte

\section{Relatividade Especial, Notação de Índices e Cálculo Tensorial}

\lettrine[lines=3, nindent=0.1em]{\capitular\addfontfeature{Color=AA0000}C}{omeçar} os estudos
em Teoria Clássica de Campos com Relatividade Especial, além de ser usual nos livros mais adotados
em cursos univesitários, é fundamental do ponto de vista da construção das ideias.


\pagebreak
















\section{Introdução ao Princípio Variacional}

\section{Lagrangiana do Campo Eletromagnético}

\end{document}
