%!TEX program = xelatex
% !TEX root = main.tex
\documentclass[12pt,a4paper]{article}

% Packages
\usepackage[brazilian]{babel}           % Fix Latex dummy fonts
\usepackage{amsmath,amssymb,amsfonts,mathtools} % Math
\usepackage{physics}               % Handy shorthand for derivatives, bras/kets, etc.
\usepackage{tensor}                % Indexed tensors
\usepackage{authblk}               % For clean multiple-author formatting
\usepackage{hyperref}              % Clickable links in PDF
\usepackage[margin=2.5cm]{geometry}
\usepackage{bookmark}              % Smarter PDF bookmark
\usepackage{graphicx}              % For figures
\usepackage{fancyhdr}              % Custom headers/footers
\usepackage{enumitem}              % Better control of lists/enumeration
\usepackage{xcolor}                % Color support
\usepackage{bm}                    % Bold math symbols
\usepackage{tikz}                  % For diagrams (optional but powerful)
\usepackage{cleveref}              % Smart cross-referencing (works well with hyperref)
\usepackage{titlesec}              % Customize section headings
\usepackage{titling}               % More control over the title
\usepackage{fontspec}
\usepackage{lettrine}
\usepackage{csquotes}
\usepackage{footnote}

%bib
\usepackage[backend=biber, style=numeric-comp, sorting=none]{biblatex}
\addbibresource{refs.bib}

%-------------------------------------Setup Visual---------------------------------------------%
\setlength{\parskip}{0.5em}       % Space between paragraphs
\setlength{\parindent}{0pt}       % No indentation
\pagestyle{fancy}
\fancyhf{}
\rhead{Notas em Teoria Clássica de Campos}
\lhead{GFTinho}
\cfoot{\thepage}
\titleformat{\section}{\large\bfseries}{\thesection.}{1em}{}
\titleformat{\subsection}{\normalsize\bfseries}{\thesubsection.}{1em}{}

% --- CONFIGURAÇÃO DA FONTE ORNAMENTADA (MÉTODO LOCAL) ---
% Usamos a opção "Path" para indicar a pasta onde a fonte está.
% O comando agora procura pelo NOME DO ARQUIVO, não pelo nome da família.
\newfontfamily{\capitular}[
  Path = ./,  % Indica a subpasta.
  Extension = .ttf % Opcional, mas boa prática.
]{Woodcut1}


\begin{document}

\begin{titlepage}

    \newcommand{\HRule}{\rule{\linewidth}{0.5mm}} % Defines a new command for the horizontal lines, change thickness here
    
    \center % Center everything on the page
    %----------------------------------------------------------------------------------------
    %	HEADING SECTIONS
    %----------------------------------------------------------------------------------------
    \textsc{\LARGE Universidade Federal de Minas Gerais}\\[1.5cm] % Name of your university/college
    \textsc{\Large Programa de Graduação em Física}\\[0.5cm] % Major heading such as course name
    \textsc{\large ICEX}\\[0.5cm] % Minor heading such as course title
    \vfill
    %----------------------------------------------------------------------------------------
    %	TITLE SECTION
    %----------------------------------------------------------------------------------------
    \HRule \\[0.4cm]
    { \huge \bfseries Notas em Teoria Clássica de Campos}\\[0.4cm] % Title of your document
    \HRule \\[1.5cm]
    %----------------------------------------------------------------------------------------
    %	AUTHOR SECTION
    %----------------------------------------------------------------------------------------
    \begin{minipage}{0.4\textwidth}
    \begin{flushleft} \large
    \emph{Autores:}\\
    Álvaro \quad Cássia \quad Henrique \quad João Monteiro \quad Mariana \quad Matheus \quad Paulo Vitor % Your names
    \end{flushleft}
    \end{minipage}
    ~
    \begin{minipage}{0.4\textwidth}
    \begin{flushright} \large
    \emph{Professor:} \\
    Filipe \textsc{Menezes} 
    \end{flushright}
    \end{minipage}\\[2cm]
    \vspace*{6cm}
    %----------------------------------------------------------------------------------------
    %	DATE SECTION
    %----------------------------------------------------------------------------------------
    {\large \today}\\[2cm] 
    %----------------------------------------------------------------------------------------
    %	LOGO SECTION
    %----------------------------------------------------------------------------------------
    %\includegraphics[width = 0.3\linewidth]{principal_ufmg.jpg}
    %----------------------------------------------------------------------------------------
\end{titlepage}

\tableofcontents % Gera o sumário
\newpage % Começa o conteúdo na página seguinte

\section{Relatividade Especial, Notação de Índices e Cálculo Tensorial}

\lettrine[lines=3, nindent=0.1em]{\capitular\addfontfeature{Color=AA0000}C}{omeçar} os estudos
em Teoria Clássica de Campos com Relatividade Especial, além de ser usual nos livros mais adotados
em cursos univesitários, é fundamental para a construção da base conceitual e do \textit{framework}
matemático necessário no tratamento de campos clássicos para as interações fundamentais.
Talvez o principal e mais famoso exemplo de um campo clássico desse tipo é o campo eletromagnético, que,
em sua natureza, é governado por leis físicas que são covariantes sob Transformações de Lorentz
(i.e., são as mesmas após realizarmos mudanças de coordenadas entre referenciais inerciais).
Começaremos agora a introduzir os conceitos mais importantes de relatividade especial para o tratamento de
campos clássicos fundamentais.

\subsection{Postulados da Relatividade Especial e Espaço-Tempo}
Enunciados por Einstein em 1905, ao tratar, justamente,
de questões fundamentais envolvendo o campo eletromagnético\cite{einstein1905}, os postulados da relatividade
especial podem ser adaptados em linguagem moderna da seguinte forma:
\begin{enumerate}[label=\Roman*.]
  \item As leis da Física são as mesmas para todos referenciais inerciais.
  \item A velocidade da luz é finita, com valor definido $c$ para todo e qualquer referencial.
\end{enumerate}

Estes são os alicerces a partir dos quais construiremos toda a teoria. Usaremos principalmente o postulado (I)
para nos guiar a respeito do estabelecimento de qualquer framework matématico que viermos a desenvolver. Para
entender a motivação por trás dos postulados, vale conferir o primeiro parágrafo em \cite{einstein1905}.

Devido aos postulados, surge de forma natural um tratamento equivalente de espaço e tempo na relatividade
especial, antes vistos como dois entes totalmente separados e independentes. Em outras palavras, devido à demanda
de que as leis da Física sejam covariantes por Lorentz, acabamos, inevitavelmente, por tratar espaço e tempo de
forma equivalente, fundindo-os no mesmo objeto que denominamos de \enquote{espaço-tempo}. 

Nessa nova forma de tratar o espaço e o tempo, de modo que as leis Físicas não mudem sob mudanças de referenciais
inerciais, acabamos introduzindo uma nova métrica, isto é, uma nova forma de \enquote{medir distâncias} nesse novo
\enquote{espaço} construído a partir dessa \enquote{fusão} entre espaço e tempo. Essa nova métrica é dada
por um escalar\footnote{\enquote{Escalar} empregado no sentido de ser um invariante por Lorentz, que significa que é um número que permanece o mesmo após mudarmos entre referenciais inerciais.}
$\mathrm{ds^2}$, que não é para ser entendido como \enquote{um número pequeno elevado ao quadrado}, mas sim 
como um objeto por si só. Em coordenadas cartesianas para o espaço-tempo, esse escalar é dado por:
\begin{equation*}
  ds^2 = -dt^2 + d\vec{x}^2
\end{equation*}
onde $d\vec{x}^2$ representa $dx^2 + dy^2 + dz^2$ para o espaço tridimensional (notação boba). Novamente,
$\mathrm{dt}$ e $\mathrm{dx}$ não representam \enquote{infinitesimais}, mas podem ser entendidos como tal a um primeiro momento;
mais adiante pretendemos esclarecer do real significado matemático que esses objetos carregam, quando formos
descrever a métrica de forma tensorial\footnote{Uma generalização de matrizes, ou melhor, de vetores; nada assustador.}.


\section{Introdução ao Princípio Variacional}

\section{Lagrangiana do Campo Eletromagnético}


\printbibliography
\end{document}
